\sect{Functional Analysis}\cite{brezis}
\subsec{Hahn-Banach Theorems}
\THM{Hahn-Banach of analytic form,hahn-banach-of-analytic-form}
Let $p \colon E \to \bb R$ be a sublinear function on a vector space $E$ (i.e., $\forall \lambda > 0, x, y \in E,\ p(\lambda x) = \lambda p(x), p(x+y) \le p(x) + p(y)$), $G \subset E$ a linear subspace, and $g \colon G \to \bb R$ a linear functional such that $\forall x \in G,\ g(x) \le p(x)$. Then, $\exists$ a linear functional $f \colon E \to \bb R$ that extends $g$ and that $\forall x \in E,\ f(x) \le p(x)$.

\DEF{Norm on the dual space of a normed space,norm-on-the-dual-space-of-a-normed-space}
For a normed space $E$, the \EMPH{dual norm} on $E^*$ is defined by
\[
  \norm{f}_{E^*} = \sup_{\substack{\norm x \le 1 \\ x \in E}} \abs{f(x)} = \sup_{\substack{\norm x \le 1 \\ x \in E}} f(x).
\]

\DEF{Scalar product for the duality,scalar-product-for-the-duality}
For a vector space $E$ and its dual space $E^*$, $\scal\ \  \colon E^* \times E \to \bb R$ defined by $\scal fx = f(x)$ is called the \EMPH{scalar product for the duality} $E$, $E^*$.

\DEF{Strictly convex normed space,strictly-convex-normed-space}
A normed space $E$ is said to be \EMPH{strictly convex} if $\forall t \in (0, 1), x, y \in E$ with $\norm x = \norm y = 1$, $\norm{tx + (1-t)y} < 1$ except for $x = y$.

\COR{Hahn-Banach of alternate form,hahn-banach-of-alternate-form}
For a continuous linear functional $g \colon G \to \bb R$ on a linear subspace $G \subset E$ of a normed space $E$, $\exists f \in E^*$ that extends $g$ and that $\norm f_{E^*} = \norm g_{G^*}$.

In the case when $G = \bb R x_0$ and $g(tx_0) = t \norm{x_0}^2$ for a given $x_0 \in E$, $\exists f_0 \in E^*$ such that $\norm{f_0} = \norm{x_0}$ and $\scal{f_0}{x_0} = \norm{x_0}^2$. If $E^*$ is strictly convex, then $f_0$ is unique.

\DEF{Duality map from a normed space into its dual space,duality-map-from-a-normed-space-into-its-dual-space}
For a normed space $E$ and $x_0 \in E$, define
\[
  F(x_0) = \{ f_0 \in E^* \mid \norm{f_0} = \norm{x_0},\ \scal{f_0}{x_0} = \norm{x_0}^2 \}.
\]
The \EMPH{duality map} from $E$ into $E^*$ is a multivalued map $x_0 \mapsto F(x_0)$.

\COR{Norm of a vector is the max of its scalar product,norm-of-a-vector-is-the-max-of-its-scalar-product}
For a normed space $E$ and $x \in E$,
\[
  \norm x = \sup_{\substack{f \in E^* \\ \norm f \le 1}} \abs{\scal fx} = \max_{\substack{f \in E^* \\ \norm f \le 1}} \abs{\scal fx}.
\]

\DEF{Affine hyperplane of a normed space,affine-hyperplane-of-a-normed-space}
For a normed space $E$ and a linear functional $f \colon E \to \bb R$, an affine \EMPH{hyperplane} is a subset $H = \{ x \in E \mid f(x) = \alpha \} \subset E$ with $\alpha \in \bb R$, written $H = [f = \alpha]$. $f = \alpha$ is called the \EMPH{equation}.

\PROP{Linear functional is continuous iff its hyperplane is closed,linear-functional-is-continuous-iff-its-hyperplane-is-closed}
For a linear functional $f \colon E \to \bb R$ on a normed space $E$ and $\alpha \in \bb R$, $[f = \alpha]$ is closed iff $f$ is continuous.

\DEF{Separation by a hyperplane,separation-by-a-hyperplane}
For two subsets $A, B \subset E$ of a normed space $E$, the hyperplane $[f = \alpha] \subset E$ \EMPH{separates} $A$ and $B$ if
\[
  \forall x \in A, y \in B,\ f(x) \le \alpha \le f(y);
\]
\EMPH{strictly separates} if
\[
  \exists \epsilon > 0, \forall x \in A, y \in B,\ f(x) \le \alpha - \epsilon < \alpha + \epsilon \le f(y).
\]

\DEF{Convex subset of a normed space,convex-subset-of-a-normed-space}
A subset $A \subset E$ of a normed space $E$ is said to be \EMPH{convex} if
\[
  \forall t \in [0, 1], x, y \in A,\ tx + (1-t)y \in A.
\]

\THM{Hahn-Banach of first geometric form,hahn-banach-of-first-geometric-form}
For two disadjoint nonempty convex subsets $A, B \subset E$ of a normed space $E$ with one of them open, $\exists$ a closed hyperplane that separates $A$ and $B$.

\DPROP{Minkowski functional of an open convex set,minkowski-functional-of-an-open-convex-set}
Let $C \subset E$ be an open convex subset of a normed space $E$ with $0 \in C$, and for $x \in E$
\[
  p(x) = \inf \{ \alpha > 0 \mid \alpha^{-1} x \in C \},
\]
called the \EMPH{gauge} or the \EMPH{Minkowski functional} of $C$.
Then, $p$ satisfies the folowing properties:
\begin{enumerate}
\item $p$ is sublinear,
\item $\exists M, \forall x \in E,\ 0 \le p(x) \le M \norm x$,
\item $C = \{ x \in E \mid p(x) < 1\}$.
\end{enumerate}

\LEM{There exists a hyperplane that separates an open convex and outside point,there-exists-a-hyperplane-that-separetes-an-open-convex-and-outside-point}
For a nonempty open convex $C \subset E$ of a normed space $E$ and $x \in E \setminus C$, $\exists f \in E^*$ such that $\forall x \in C,\ f(x) < f(x_0)$. In particular, the hyperplane $[f = f(x_0)]$ separetes $\{ x_0 \}$ and $C$.

\THM{Hahn-Banach of second geometric form,hahn-banach-of-second-geometric-form}
For two disadjoint nonempty convex subsets $A, B \subset E$ of a normed space $E$ with $A$ closed and $B$ compact, $\exists$ a closed hyperplane that strictly separates $A$ and $B$.

\COR{Some linear functional can vanish on a linear subspace,some-linear-functional-can-vanish-on-a-linear-subspace}
For a linear subspace $F \subset E$ of a normed space $E$ with $\BAR F \neq E$, $\exists f \in E^*$ such that
\[
  \forall x \in F,\ \scal fx = 0,\quad f \not\equiv 0.
\]

\DEF{Notation of a bidual space,notation-of-a-bidual-space}
Let $E$ be a normed space, and $J \colon E \ni x \mapsto Jx \in E^{**}$ a \EMPH{canonical injection} (i.e., $Jx \colon f \mapsto \scal fx$, or $\scal{Jx}{f} = \scal fx$). Then, $J$ is an \EMPH{isometry}:
\[
  \norm{Jx}_{E^{**}} = \sup_{\substack{f \in E^* \\ \norm f \le 1}} \abs{\scal{Jx}{f}} = \sup_{\substack{f \in E^* \\ \norm f \le 1}} \abs{\scal fx} = \norm x_E.
\]

$J$ can be not surjective; if $J$ is surjective, $E$ is said to be \EMPH{reflexive}.

\DEF{Orthogonal complement,orthogonal-complement}
For a linear subspace $M \subset E$ of a normed space $E$ and a linear subspace $N \subset E^*$, their \EMPH{orthogonal complements} are
\begin{align*}
  M^\perp &= \{ f \in E^* \mid \forall x \in M,\ \scal fx = 0\} \subset E^* \\
  N^\perp &= \{ x \in E \mid \forall f \in N,\ \scal fx = 0\} \subset E,
\end{align*}
respectively.

\PROP{Relation between a linear subspace and its orthogonal complement,relation-between-a-linear-subspace-and-its-orthogonal-complement}
For a linear subspace $M \subset E$ of a normed space $E$ and a linear subspace $N \subset E^*$,
\[
  (M^\perp)^\perp = \BAR M,\quad (N^\perp)^\perp \supset \BAR N.
\]

If $E$ is reflexive, then $(N^\perp)^\perp = \BAR N$.
