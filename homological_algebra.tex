\section{Homological Algebra}\cite{weibel}
\subsection{Chain Complexes}
\DEF{Chain complex of R-modules,chain-complex-of-r-modules}
A \EMPH{chain complex} $C_\cdot$ of $R$-modules is a family $\{ C_n \}_{n \in \bb Z}$ of $R$-modules with $R$-module maps $d_n \colon C_n \to C_{n-1}$ such that $d_{n-1} \circ d_{n} = 0$, called the \EMPH{differential}. $Z_n = Z_n(C_\cdot) \defeq \ker d_n$ is the module of $n$-\EMPH{cycles} of $C_\cdot$, and $B_n = B_n(C_\cdot) \defeq \im d_{n+1}$ is the module of $n$-\EMPH{boundaries} of $C_\cdot$. The $n$-th \EMPH{homology module} of $C_\cdot$ is $H_n(C_\cdot) \defeq Z_n / B_n$.

A \EMPH{chain complex map} $u \colon C_\cdot \to D_\cdot$ is a family of $R$-module homomorphisms $u_n \colon C_n \to D_n$ commuting with $d$ (i.e., $u_{n-1}d_n = d_n u_n$).

\DPROP{A category of chain complexes of R-modules,a-category-of-chain-complexes-of-r-modules}
$\ch(\modb R)$ is a category whose objects are chain complexes of (right) $R$-modules and whose arrows are chain complex maps. An arrow $u \colon C_\cdot \to D_\cdot$ of $\ch(\modb R)$ sends boundaries to boundaries and cycles to cycles. $H_n \colon \ch(\modb R) \to \modb R$ is a functor.

\PROP{Split exact sequence of vector spaces,split-exact-sequence-of-vector-spaces}
For a family $\{ B_n, H_n \}$ of vector spaces, $\{ C_n = B_n \oplus H_n \oplus B_{n-1} \}$ with projection-inclusions $\pi_n \colon C_n \to B_{n-1} \subset C_{n-1}$ is a chain complex. Every chain complex of vector spaces is isomorphic to such a complex.

\DEF{Quasi-isomorphism between chain complexes,quasi-isomorphism-between-chain-complexes}
A chain map $C_\cdot \to D_\cdot$ between chain complexes $C_\cdot, D_\cdot$ is called a \EMPH{quasi-isomorphism} if the maps $H_n(C_\cdot) \to H_n(D_\cdot)$ are all isomorphisms.

\DEF{Cochain complex of R-modules,cochain-complex-of-r-modules}
A \EMPH{cochain complex} $C^\cdot$ of $R$-modules is a family $\{C^n\}$ of $R$-modules with $R$-module maps $d^n \colon C^n \to C^{n+1}$ such that $d^{n+1}d^n = 0$. $Z^n(C^\cdot) \defeq \ker d^n$ is the module of $n$-\EMPH{cocycles}, and $B^n(C^\cdot) \defeq \im d^{n-1}$ is the module of $n$-\EMPH{coboundaries}. The $n$-th \EMPH{cohomology module} of $C^\cdot$ is $H^n(C^\cdot) \defeq Z^n/B^n$.
