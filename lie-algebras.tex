\newcommand{\gl}{\g{gl}}
\renewcommand{\sl}{\g{sl}}
\renewcommand{\o}{\g{o}}
\section{Lie Algebra\cite{samu}}
\subsection{Fundations}
\subsubsection{Lie algebra [Definition \ref{lie-algebra}]}\label{lie-algebra}
A vector space $\g g$ over a field $\bb{K}$ with the Lie bracket satisfying the conditions
\begin{enumerate}
\item Lie bracket is bilinear
\item $\forall x \in \g g,\ [x,x] = 0$
\item $\forall x, y, z \in \g g,\ [[x,y],z]+[[y,z],x]+[[z,x],y] = 0$
\end{enumerate}
is called a \EMPH{Lie algebra} over $\bb{K}$.

\subsubsection{General linear Lie algebra [Definition \ref{general-linear-lie-algebra}]}\label{general-linear-lie-algebra}
$\g{gl}_n(\bb{R})$ is the Lie algebra $M_n(\bb{R})$ with the Lie bracket $[x,y] = xy - yx$.

\subsubsection{Derivation algebra [Definition \ref{derivation-algebra}]}\label{derivation-algebra}
A linear endomorphism $D$ of an algebra $\bb{A}$ over $\bb{R}$ satisfying $D(xy) = D(x)y + xD(y)$ is called a \EMPH{derivation} of $\bb{A}$. The set of all derivations $\mathrm{Der}\, \bb{A}$ with the addition, scaler multiplication, and lie bracket defined as follows:
\begin{enumerate}
\item $(D+D')(x) = D(x)+D'(x)$
\item $(\alpha D)(x) = \alpha D(x)$
\item $[D,D'](x) = D(D'(x)) - D'(D(x))$
\end{enumerate}
is a Lie algebra called the \EMPH{derivation algebra} of $\bb{A}$.

\subsubsection{Lie subalgebra [Definition \ref{lie-subalgebra}]}\label{lie-subalgebra}
A linear subspace $\g h \subset \g g$ of a Lie algebra $\g g$ is a \EMPH{Lie subalgebra} of $\g g$ if $\forall x, y \in \g h$, $[x,y] \in \g h$.

For linear subspaces $\g a, \g b \subset \g g$, $[\g a, \g b]$ denotes the subspace generated by $[x,y]$ with $x \in \g a, y \in \g b$.

\subsubsection{Special linear Lie algebra [Definition \& Proposition \ref{special-linear-lie-algebra}]}\label{special-linear-lie-algebra}
$\sl_n(\bb{R}) = \{ x \in \gl_n(\bb{R}) \mid \tr x = 0 \} $ is a Lie subalgebra of $\gl_n(\bb{R})$.

\subsubsection{Orthogonal Lie algebra [Definition \& Proposition \ref{orthogonal-lie-algebra}]}\label{orthogonal-lie-algebra}
$\o(n) = \{ x \in \gl_n(\bb{R}) \mid {}^tx = -x \}$ is a Lie subalgebra of $\sl_n(\bb{R})$.

\subsubsection{Ideal of a Lie algebra [Definition \& Proposition \ref{ideal-of-a-lie-algebra}]}\label{ideal-of-a-lie-algebra}
A linear subspace $\g h \subset \g g$ of a Lie algebra $\g g$ is an \EMPH{ideal} of $\g g$ if $\forall x \in \g g, y \in \g h,\ [x,y] \in \g h$.

For ideals $\g a, \g b \subset \g g$, $[\g a, \g b]$ is also an ideal.

\subsubsection{Derived ideal of a Lie algebra [Definition \ref{derived-ideal-of-a-lie-algebra}]}\label{derived-ideal-of-a-lie-algebra}
For a Lie algebra $\g g$, $D\g g = [\g g, \g g]$ is an ideal of $\g g$ called the \EMPH{derived ideal} of $\g g$.

If $\g g = \gl_n(\bb{R})$, $D\g g = \sl_n(\bb{R})$.

\subsubsection{Homomorphism of Lie algebras [Definition \& Proposition \ref{homomorphism-of-lie-algebras}]}\label{homomorphism-of-lie-algebras}
For Lie algebras $\g g, \g h$, a linear map $\varphi \colon \g g \to \g h$ is called a \EMPH{homomorphism} if $\forall x, y \in \g g,\ \varphi([x,y]) = [\varphi(x), \varphi(y)]$. A homomorphism $\varphi$ is an \EMPH{isomorphism} if it is bijective. Lie algebras between which there exists an isomorphism are said to be \EMPH{isomorphic} to each other, written as $\g g \cong \g h$.

A composite of homomorphisms is also a homomorphism, and that of isomorphisms is also an isomorphism.

The kernel $\ker \varphi = \{ x \in \g g \mid \varphi(x) = 0 \}$ of a homomorphism $\varphi$ is an ideal of $\g g$ while the image $\im \varphi = \varphi(\g g)$ of $\varphi$ is a Lie subalgebra of $\g h$.

\subsubsection{Representation of a Lie algebra on a vector space [Definition \ref{representation-of-a-lie-algebra-on-a-vector-space}]}\label{representation-of-a-lie-algebra-on-a-vector-space}
For a Lie algebra $\g g$ and a vector space $V$, a homomorphism $\rho \colon \g g \to \g{gl}(V)$ is called a \EMPH{representation} of $\g g$ on $V$.

\subsubsection{Adjoint representation of a Lie algebra [Definition \& Proposition \ref{adjoint-representation-of-a-lie-algebra}]}\label{adjoint-representation-of-a-lie-algebra}
For a Lie algebra $\g g$ and $x \in \g g$, define a derivation $\ad (x) \colon \g g \to \g g$ by $\ad(x)(y) = [x,y]$. A representation $\ad \colon \g g \ni x \mapsto \ad(x) \in \gl (\g g)$ is called the \EMPH{adjoint representation} of $\g g$. The \EMPH{center} of $\g g$ is $\g z = \ker(\ad)$, which is a commutative ideal. $\im(\ad)$ is an ideal of $\der \g g$. A derivation $\ad(x)$ is called a \EMPH{inner derivation} of $\g g$.

\subsubsection{Quotient algebra for Lie algebras [Definition \ref{quotient-algebra-for-lie-algebras}]}\label{quotient-algebra-for-lie-algebras}
For a Lie algebra $\g g$ and an ideal $\g a \subset \g g$, the \EMPH{quotient algebra} is
\[
\g g / \g a = \{ \bar x = x + \g a \mid x \in \g g \}
\]
with canonical operations, where $\bar x = \{ y \in \g g \mid x \equiv y \pmod{\g a} \} = \{ x + a \mid a \in \g a \}$ called the \EMPH{class} of $x$. The homomorphism $\varphi \colon \g g \ni x \mapsto \bar x \in \g g / \g a$ is called the \EMPH{canonical homomorphism}.

\subsubsection{The first isomorphism theorem for Lie algebras [Theorem \ref{the-first-isomorphism-theorem-for-lie-algebras}]}\label{the-first-isomorphism-theorem-for-lie-algebras}
For Lie algebras $\g g$, $\g h$ and a homomorphism $\varphi \colon \g g \to \g h$,
\[
\g g / \ker \varphi \cong \im \varphi.
\]

\subsubsection{The second isomorphism theorem for Lie algebras [Theorem \ref{the-second-isomorphism-theorem-for-lie-algebras}]}\label{the-second-isomorphism-theorem-for-lie-algebras}
For a Lie algebra $\g g$, an ideal $\g a \subset \g g$, a Lie subalgebra $\g h \subset \g g$ and the canonical homomorphism $\varphi \colon \g g \to \g g / \g a$,
\[
\g h / (\g h \cap \g a) \cong (\g h + \g a)/\g a.
\]