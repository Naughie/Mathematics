\newcommand{\gl}{\g{gl}}
\renewcommand{\sl}{\g{sl}}
\renewcommand{\o}{\g{o}}
\sect{Lie Algebra}\cite{samu}
\subsec{Foundations}
\DEF{Lie algebra,lie-algebra}
A vector space $\g g$ over a field $\bb{K}$ with the Lie bracket satisfying the conditions
\begin{enumerate}
\item Lie bracket is bilinear
\item $\forall x \in \g g,\ [x,x] = 0$
\item $\forall x, y, z \in \g g,\ [[x,y],z]+[[y,z],x]+[[z,x],y] = 0$
\end{enumerate}
is called a \EMPH{Lie algebra} over $\bb{K}$.

\DEF{General linear Lie algebra,general-linear-lie-algebra}
$\g{gl}_n(\bb{R})$ is the Lie algebra $M_n(\bb{R})$ with the Lie bracket $[x,y] = xy - yx$.

\DEF{Derivation algebra,derivation-algebra}
A linear endomorphism $D$ of an algebra $\bb{A}$ over $\bb{R}$ satisfying $D(xy) = D(x)y + xD(y)$ is called a \EMPH{derivation} of $\bb{A}$. The set of all derivations $\mathrm{Der}\, \bb{A}$ with the addition, scaler multiplication, and lie bracket defined as follows:
\begin{enumerate}
\item $(D+D')(x) = D(x)+D'(x)$
\item $(\alpha D)(x) = \alpha D(x)$
\item $[D,D'](x) = D(D'(x)) - D'(D(x))$
\end{enumerate}
is a Lie algebra called the \EMPH{derivation algebra} of $\bb{A}$.

\DEF{Lie subalgebra,lie-subalgebra}
A linear subspace $\g h \subset \g g$ of a Lie algebra $\g g$ is a \EMPH{Lie subalgebra} of $\g g$ if $\forall x, y \in \g h$, $[x,y] \in \g h$.

For linear subspaces $\g a, \g b \subset \g g$, $[\g a, \g b]$ denotes the subspace generated by $[x,y]$ with $x \in \g a, y \in \g b$.

\DPROP{Special linear Lie algebra,special-linear-lie-algebra}
$\sl_n(\bb{R}) = \{ x \in \gl_n(\bb{R}) \mid \tr x = 0 \} $ is a Lie subalgebra of $\gl_n(\bb{R})$.

\DPROP{Orthogonal Lie algebra,orthogonal-lie-algebra}
$\o(n) = \{ x \in \gl_n(\bb{R}) \mid {}^t\!x = -x \}$ is a Lie subalgebra of $\sl_n(\bb{R})$.

\DPROP{Ideal of a Lie algebra,ideal-of-a-lie-algebra}
A linear subspace $\g h \subset \g g$ of a Lie algebra $\g g$ is an \EMPH{ideal} of $\g g$ if $\forall x \in \g g, y \in \g h,\ [x,y] \in \g h$.

For ideals $\g a, \g b \subset \g g$, $[\g a, \g b]$ is also an ideal.

\DEF{Derived ideal of a Lie algebra,derived-ideal-of-a-lie-algebra}
For a Lie algebra $\g g$, $D\g g = [\g g, \g g]$ is an ideal of $\g g$ called the \EMPH{derived ideal} of $\g g$.

If $\g g = \gl_n(\bb{R})$, $D\g g = \sl_n(\bb{R})$.

\DPROP{Homomorphism of Lie algebras,homomorphism-of-lie-algebras}
For Lie algebras $\g g, \g h$, a linear map $\varphi \colon \g g \to \g h$ is called a \EMPH{homomorphism} if $\forall x, y \in \g g,\ \varphi([x,y]) = [\varphi(x), \varphi(y)]$. A homomorphism $\varphi$ is an \EMPH{isomorphism} if it is bijective. Lie algebras between which there exists an isomorphism are said to be \EMPH{isomorphic} to each other, written $\g g \cong \g h$.

A composite of homomorphisms is also a homomorphism, and that of isomorphisms is also an isomorphism.

The kernel $\ker \varphi = \{ x \in \g g \mid \varphi(x) = 0 \}$ of a homomorphism $\varphi$ is an ideal of $\g g$ while the image $\im \varphi = \varphi(\g g)$ of $\varphi$ is a Lie subalgebra of $\g h$.

\DEF{Representation of a Lie algebra on a vector space,representation-of-a-lie-algebra-on-a-vector-space}
For a Lie algebra $\g g$ and a vector space $V$, a homomorphism $\rho \colon \g g \to \g{gl}(V)$ is called a \EMPH{representation} of $\g g$ on $V$.

\DPROP{Adjoint representation of a Lie algebra,adjoint-representation-of-a-lie-algebra}
For a Lie algebra $\g g$ and $x \in \g g$, define a derivation $\ad (x) \colon \g g \to \g g$ by $\ad(x)(y) = [x,y]$. A representation $\ad \colon \g g \ni x \mapsto \ad(x) \in \gl (\g g)$ is called the \EMPH{adjoint representation} of $\g g$. The \EMPH{center} of $\g g$ is $\g z = \ker(\ad)$, which is a commutative ideal. $\im(\ad)$ is an ideal of $\der \g g$. A derivation $\ad(x)$ is called a \EMPH{inner derivation} of $\g g$.

\DEF{Quotient algebra for Lie algebras,quotient-algebra-for-lie-algebras}
For a Lie algebra $\g g$ and an ideal $\g a \subset \g g$, the \EMPH{quotient algebra} is
\[
\g g / \g a = \{ \BAR x = x + \g a \mid x \in \g g \}
\]
with canonical operations, where $\BAR x = \{ y \in \g g \mid x \equiv y \pmod{\g a} \} = \{ x + a \mid a \in \g a \}$ called the \EMPH{class} of $x$. The homomorphism $\varphi \colon \g g \ni x \mapsto \BAR x \in \g g / \g a$ is called the \EMPH{canonical homomorphism}.

\THM{The first isomorphism theorem for Lie algebras,the-first-isomorphism-theorem-for-lie-algebras}
For Lie algebras $\g g$, $\g h$ and a homomorphism $\varphi \colon \g g \to \g h$,
\[
\g g / \ker \varphi \cong \im \varphi.
\]

\THM{The second isomorphism theorem for Lie algebras,the-second-isomorphism-theorem-for-lie-algebras}
For a Lie algebra $\g g$, an ideal $\g a \subset \g g$, a Lie subalgebra $\g h \subset \g g$ and the canonical homomorphism $\varphi \colon \g g \to \g g / \g a$,
\[
\g h / (\g h \cap \g a) \cong (\g h + \g a)/\g a.
\]

\subsec{Solvable and Nilpotent Lie algebra}
\DEF{Solvable Lie algebra,solvable-lie-algebra}
Let $\g g$ be a Lie algebra, and
\[
D^0\g g = \g g,\quad D^k \g g = D(D^{k-1}\g g),\quad k = 1, 2, \dotsc
\]
$\g g$ is said to be \EMPH{solvable} if $D^r\g g = \{ 0 \}$ for some $r$ called the \EMPH{length} of $\g g$.

\EX{Lie algebra of triangular matrices is solvable,lie-algebra-of-triangular-matrices-is-solvable}
Let
\begin{align*}
\g g_0 &= \{ \xi = (\xi_{ij}) \in \gl_n(\bb{R}) \mid \xi\ \text{is upper triangular} \}, \\
\g g_k &= \{ \xi = (\xi_{ij}) \in \gl_n(\bb{R}) \mid \xi_{ij} = 0 \ \text{for}\ j - i < k \}.
\end{align*}
Then, $[\g g_0, \g g_0] \subset \g g_1$, $[\g g_k, \g g_\ell] \subset \g g_{k+\ell}$, $k, \ell = 0, 1, \dotsc$, and $\g g_0$ is a solvable Lie algebra of length $\le n$.

\THM{Lie subalgebra of a solvable Lie algebra is also solvable,lie-subalgebra-of-a-solvable-lie-algebra-is-also-solvable}
For a solvable Lie algebra $\g g$, its Lie subalgebra $\g h \subset \g g$ is also solvable, and if $\g h$ is an ideal, $\g g / \g h$ is also solvable.

\THM{Lie algebra whose ideal and quotient algebra over it are solvable is solvable,lie-algebra-whose-ideal-and-quotient-algebra-over-it-are-solvable-is-solvable}
For a Lie algebra $\g g$ and its ideal $\g a \subset \g g$, if $\g a$ and $\g g / \g a$ are both solvable, then $\g g$ is also solvable.

\DEF{Nilpotent Lie algebra,nilpotent-lie-algebra}
Let $\g g$ be a Lie algebra, and
\[
C^0 \g g = \g g,\quad C^k \g g = [\g g, C^{k-1}\g g],\quad k = 1, 2, \dotsc
\]
$\g g$ is said to be \EMPH{nilpotent} if $C^s \g g = \{ 0 \}$ for some $s$ called the \EMPH{length} of $\g g$.

Since $D^k \g g \subset C^k \g g$, a nilpotent Lie algebra is solvable.

\EX{Lie algebra of strictly triangular matrices is nilpotent,lie-algebra-of-strictly-triangular-matrices-is-nilpotent}
$\g g_1$ in \ref{lie-algebra-of-triangular-matrices-is-solvable} is nilpotent while $\g g_0$ there is not.

\THM{Lie subalgebra of a nilpotent Lie algebra is also nilpotent,lie-subalgebra-of-a-nilpotent-lie-algebra-is-also-nilpotent}
For a nilpotent Lie algebra $\g g$, its Lie subalgebra $\g h \subset \g g$ is also nilpotent, and if $\g h$ is an ideal, $\g g/ \g h$ is also nilpotent.

\THM{Center of a nilpotent Lie algebra has a nonzero vector,center-of-a-nilpotent-lie-algebra-has-a-nonzero-vector}
For a Lie algebra $\g g$ and its center $\g z$, $\g z \neq \{ 0 \}$ if $\g g$ is nilpotent while $\g g$ is nilpotent if $\g g / \g z$ is nilpotent.
