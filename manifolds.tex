\section{Manifolds}
\subsection{Manifolds on Euclidean Spaces}
\subsubsection{Taylor's theorem with remainder [Theorem \ref{taylors-theorem-with-remainder}]}\label{taylors-theorem-with-remainder}
A smooth function $f$ on an open ball $U \in \mathcal{O}_n$ can be written as
\[
f(x) = f(p) + \sum (x^i - p^i) g_i(x)
\]
where $p \in U$ and $g_i \in C^\infty (U)$ with $g_i(p) = (\partial f / \partial x^i)(p)$.

Adapting this to $g_i$ repeatedly gives the Taylor's expansion of $f$.

\subsubsection{Tangent vector as an arrow from a point [Definition \ref{tangent-vector-as-an-arrow-from-a-point}]}\label{tangent-vector-as-an-arrow-from-a-point}
The \textit{tangent space} $T_p (\bb{R}^n)$ at $p \in \bb{R}^n$ is the set of arrows from $p$.

\subsubsection{Directional derivative [Definition \ref{directional-derivative}]}\label{directional-derivative}
The \textit{directional derivative} of a smooth function $f$ in the direction $v \in T_p(\bb{R}^n)$ at $p \in \bb{R}^n$ is
\[
D_v f = \lim_{t \to 0} \frac{f(c(t)) - f(p)}{t} = \left. \frac{d}{dt}\right|_{t=0} f(c(t))
\]
with $c^i(t) = p^i + tv^i$.

By the chain rule,
\[
D_v f = \sum \frac{dc^i}{dt}(0)\pfrac{f}{x^i}(p) = \sum v^i \pfrac{f}{x^i}(p).
\]

\subsubsection{Derivation at a point [Definition \& Proposition \ref{derivation-at-a-point}]}\label{derivation-at-a-point}
A linear map $D \colon C_p^\infty \to \bb{R}$ satisfying the Leibniz rule (i.e., $D(fg) = (D f)g(p) + f(p)D g$ for any $f, g \in C_p^\infty$) is called a \textit{derivation at} $p$ or a \textit{point-derivation} of $C_p^\infty$.

The set of all derivations at $p$ $\mathcal{D}_p(\bb{R}^n)$ is a real vector space, and a map $\phi \colon T_p(\bb{R}^n) \to \mathcal{D}_p(\bb{R}^n)$ assigning $D_v$ to each $v$ is a linear map.

\subsubsection{Point-derivation of a constant is zero [Lemma \ref{point-derivation-of-a-constant-is-zero}]}\label{point-derivation-of-a-constant-is-zero}
If $D$ is a point-derivation of $C_p^\infty$, then $D(c) = 0$ for any constant function $c$.

\subsubsection{Tangent space is isomorphic to the set of point-derivations [Theorem \ref{tangent-space-is-isomorphic-to-the-set-of-point-derivations}]}\label{tangent-space-is-isomorphic-to-the-set-of-point-derivations}
The linear map $\phi \to T_p(\bb{R}^n) \to \mathcal{D}_p(\bb{R}^n)$ in \ref{derivation-at-a-point} is an isomorphism of vector spaces.

\subsubsection{Tangent vector as a derivation [Definition \ref{tangent-vector-as-a-derivation}]}\label{tangent-vector-as-a-derivation}
By \ref{tangent-space-is-isomorphic-to-the-set-of-point-derivations}, $v \in T_p(\bb{R}^n)$ is identified as
\[
v = \sum v^i \left.\pfrac{}{x^i}\right|_p \in \mathcal{D}_p(\bb{R}^n).
\]

\subsubsection{Vector fields on an open set [Definition \ref{vector-fields-on-an-open-set}]}\label{vector-fields-on-an-open-set}
A \textit{vector field} on $U \in \mathcal{O}_n$ is a map $X \colon U \to T_p(\bb{R}^n)$. $X = \sum a^i \partial / \partial x^i$ means
\[
X(p) = X_p = \sum a^i(p) \left. \pfrac{}{x^i} \right|_p \quad \text{with } a^i (p) \in \bb{R}
\]
$X$ is said to be $C^\infty$ if all $a^i$s are $C^\infty$ on $U$. The set of all smooth vector fields on $U$ is denoted by $\g X(U)$.

\subsubsection{Multiplication of a smooth vector field and function [Definition \& Proposition \ref{multiplication-of-a-smooth-vector-field-and-function}]}\label{multipliaction-of-a-smooth-vector-field-and-function}
For $X \in \g X(U)$ and $f \in C^\infty(U)$, define $fX \in \g X(U)$ and $Xf \in C^\infty(U)$ as follows:
\begin{align*}
(fX)_p &= f(p)X_p = \sum (f(p)a^i(p)) \left. \pfrac{}{x^i}\right|_p,\\
(Xf)(p) &= X_p f = \sum a^i (p) \pfrac{f}{x^i} (p).
\end{align*}

\subsubsection{Leibniz rule for a vector field [Proposition \ref{leibniz-rule-for-a-vector-field}]}\label{leibniz-rule-for-a-vector-field}
For any $X \in \g X(U), f, g \in C^\infty(U)$,
\[
X(fg) = (Xf)g + fXg.
\]

\subsubsection{Derivations one-to-one-correspond to smooth vector fields [Proposition \ref{derivations-one-to-one-correspond-to-smooth-vector-fields}]}\label{derivations-one-to-one-correspond-to-smooth-vector-fields}
$\varphi \colon \g X(U) \ni X \mapsto (f \mapsto Xf) \in \mathrm{Der}\, (C^\infty(U))$ is an linear isomorphism.

\subsubsection{k-tensor on a vector space [Definition \ref{k-tensor-on-a-vector-space}]}\label{k-tensor-on-a-vector-space}
A $k$-linear function on a vector space $V$ $f \colon V^k \to \bb{R}$ is called a \textit{k-tensor} on $V$. The vector space of all $k$-tensors on $V$ is denoted by $L_k(V)$. $k$ is called the degree of $f$.

\subsubsection{Symmetric and alternating k-tensor [Definition \ref{symmetric-and-alternating-k-tensor}]}\label{symmetric-and-alternating-k-tensor}
A $k$-tensor $f \colon V^k \to \bb{R}$ is \textit{symmetric} if
\[
\forall \sigma \in \g S_k,\ f(v_{\sigma(1)},\dotsc, v_{\sigma(k)}) = f(v_1,\dotsc,v_k),
\]
and $f$ is \textit{alternating} if
\[
\forall \sigma \in \g 1S_k,\ f(v_{\sigma(1)},\dotsc, v_{\sigma(k)}) = (\sgn \sigma) f(v_1,\dotsc, v_k).
\]

\subsubsection{The set of all alternating k-tensors [Definition \ref{the-set-of-all-alternating-k-tensor}]}\label{the-set-of-all-alternating-k-tensor}
An alternating $k$-tensor on a vector space $V$ is also called a \textit{k-covector} or a \textit{multicovector of degree k} on $V$. The set of all $k$-covectors on $V$ is denoted by $A_k(v)$ for $k > 0$; for $k = 0$, $A_0 (V) = \bb{R}$.