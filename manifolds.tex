\section{Manifolds\cite{loring}}
\subsection{Manifolds on Euclidean Spaces}
\subsubsection{Taylor's theorem with remainder [Theorem \ref{taylors-theorem-with-remainder}]}\label{taylors-theorem-with-remainder}
A smooth function $f$ on an open ball $U \in \usualtop$ can be written as
\[
f(x) = f(p) + \sum (x^i - p^i) g_i(x)
\]
where $p \in U$ and $g_i \in C^\infty (U)$ with $g_i(p) = (\partial f / \partial x^i)(p)$.

Adapting this to $g_i$ repeatedly gives the Taylor's expansion of $f$.

\subsubsection{Tangent vector as an arrow from a point [Definition \ref{tangent-vector-as-an-arrow-from-a-point}]}\label{tangent-vector-as-an-arrow-from-a-point}
The \EMPH{tangent space} $T_p (\bb{R}^n)$ at $p \in \bb{R}^n$ is the set of arrows from $p$.

\subsubsection{Directional derivative [Definition \ref{directional-derivative}]}\label{directional-derivative}
The \EMPH{directional derivative} of a smooth function $f$ in the direction $v \in T_p(\bb{R}^n)$ at $p \in \bb{R}^n$ is
\[
D_v f = \lim_{t \to 0} \fr{f(c(t)) - f(p)}{t} = \left. \fr{d}{dt}\right|_{t=0} f(c(t))
\]
with $c^i(t) = p^i + tv^i$.

By the chain rule,
\[
D_v f = \sum \fr{dc^i}{dt}(0)\pfr{f}{x^i}(p) = \sum v^i \pfr{f}{x^i}(p).
\]

\subsubsection{Derivation at a point [Definition \& Proposition \ref{derivation-at-a-point}]}\label{derivation-at-a-point}
A linear map $D \colon C_p^\infty \to \bb{R}$ satisfying the Leibniz rule (i.e., $D(fg) = (D f)g(p) + f(p)D g$ for any $f, g \in C_p^\infty$) is called a \EMPH{derivation} at $p$ or a \EMPH{point-derivation} of $C_p^\infty$.

The set of all derivations at $p$ denoted by $\mathcal{D}_p(\bb{R}^n)$ is a real vector space, and a map $\phi \colon T_p(\bb{R}^n) \to \mathcal{D}_p(\bb{R}^n)$ assigning $D_v$ to each $v$ is a linear map.

\subsubsection{Point-derivation of a constant is zero [Lemma \ref{point-derivation-of-a-constant-is-zero}]}\label{point-derivation-of-a-constant-is-zero}
If $D$ is a point-derivation of $C_p^\infty$, then $D(c) = 0$ for any constant function $c$.

\subsubsection{Tangent space is isomorphic to the set of point-derivations [Theorem \ref{tangent-space-is-isomorphic-to-the-set-of-point-derivations}]}\label{tangent-space-is-isomorphic-to-the-set-of-point-derivations}
The linear map $\phi \to T_p(\bb{R}^n) \to \mathcal{D}_p(\bb{R}^n)$ in \ref{derivation-at-a-point} is an isomorphism of vector spaces.

\subsubsection{Tangent vector as a derivation [Definition \ref{tangent-vector-as-a-derivation}]}\label{tangent-vector-as-a-derivation}
By \ref{tangent-space-is-isomorphic-to-the-set-of-point-derivations}, $v \in T_p(\bb{R}^n)$ is identified as
\[
v = \sum v^i \left.\pfr{}{x^i}\right|_p \in \mathcal{D}_p(\bb{R}^n).
\]

\subsubsection{Vector fields on an open set [Definition \ref{vector-fields-on-an-open-set}]}\label{vector-fields-on-an-open-set}
A \EMPH{vector field} on $U \in \usualtop$ is a map $X \colon U \to T_p(\bb{R}^n)$. $X = \sum a^i \partial / \partial x^i$ means
\[
X(p) = X_p = \sum a^i(p) \left. \pfr{}{x^i} \right|_p \quad \text{with } a^i (p) \in \bb{R}
\]
$X$ is said to be $C^\infty$ if all $a^i$s are $C^\infty$ on $U$. The set of all smooth vector fields on $U$ is denoted by $\g X(U)$.

\subsubsection{Multiplication of a smooth vector field and function [Definition \& Proposition \ref{multiplication-of-a-smooth-vector-field-and-function}]}\label{multiplication-of-a-smooth-vector-field-and-function}
For $X \in \g X(U)$ and $f \in C^\infty(U)$, define $fX \in \g X(U)$ and $Xf \in C^\infty(U)$ as follows:
\begin{align*}
(fX)_p &= f(p)X_p = \sum (f(p)a^i(p)) \left. \pfr{}{x^i}\right|_p,\\
(Xf)(p) &= X_p f = \sum a^i (p) \pfr{f}{x^i} (p).
\end{align*}

\subsubsection{Leibniz rule for a vector field [Proposition \ref{leibniz-rule-for-a-vector-field}]}\label{leibniz-rule-for-a-vector-field}
For any $X \in \g X(U), f, g \in C^\infty(U)$,
\[
X(fg) = (Xf)g + fXg.
\]

\subsubsection{Derivations one-to-one-correspond to smooth vector fields [Proposition \ref{derivations-one-to-one-correspond-to-smooth-vector-fields}]}\label{derivations-one-to-one-correspond-to-smooth-vector-fields}
$\varphi \colon \g X(U) \ni X \mapsto (f \mapsto Xf) \in \mathrm{Der}\, (C^\infty(U))$ is an linear isomorphism.

\subsubsection{k-tensor on a vector space [Definition \ref{k-tensor-on-a-vector-space}]}\label{k-tensor-on-a-vector-space}
A $k$-linear function on a vector space $V$ $f \colon V^k \to \bb{R}$ is called a \EMPH{k-tensor} on $V$. The vector space of all $k$-tensors on $V$ is denoted by $L_k(V)$. $k$ is called the degree of $f$.

\subsubsection{Permutation action on k-tensors [Definition \ref{permutation-action-on-k-tensors}]}\label{permutation-action-on-k-tensors}
For $f \in L_k(V)$ on a vector space $V$ and $\sigma \in \g S_n$, an action of $\sigma $ on $f$ is defined by
\[
(\sigma f)(v_1, \dotsc, v_k) = f(v_{\sigma(1)},\dotsc, v_{\sigma(k)}).
\]

\subsubsection{Symmetric and alternating k-tensor [Definition \ref{symmetric-and-alternating-k-tensor}]}\label{symmetric-and-alternating-k-tensor}
A $k$-tensor $f \colon V^k \to \bb{R}$ is \EMPH{symmetric} if
\[
\forall \sigma \in \g S_k,\ \sigma f  =f,
\]
and $f$ is \EMPH{alternating} if
\[
\forall \sigma \in \g S_k,\ \sigma f = (\sgn \sigma) f.
\]

\subsubsection{The set of all alternating k-tensors [Definition \ref{the-set-of-all-alternating-k-tensor}]}\label{the-set-of-all-alternating-k-tensor}
An alternating $k$-tensor on a vector space $V$ is also called a \EMPH{k-covector} or a \EMPH{multicovector of degree k} on $V$. The set of all $k$-covectors on $V$ is denoted by $A_k(v)$ for $k > 0$; for $k = 0$, $A_0 (V) = \bb{R}$.

\subsubsection{Symmetrizing and alternating operators on k-covectors [Definition \& Proposition \ref{symmetrizing-and-alternating-operators-on-k-covectors}]}\label{symmetrizing-and-alternating-operators-on-k-covectors}
For a $f \in A_k(V)$ on a vector space $V$,
\[
 Sf = \sum_{\sigma \in \g S_n} \sigma f
\]
is symmetric, and
\[
 Af = \sum_{\sigma \in \g S_n} (\sgn \sigma) \sigma f
\]
is alternating.

\subsubsection{Tensor product of two multilinear functions [Definition \ref{tensor-product-of-two-multilinear-functions}]}\label{tensor-product-of-two-multilinear-functions}
For $f \in L_k(V), g \in L_\ell(V)$ on a vector space $V$, the \EMPH{tensor product} $f \otimes g \in L_{k+\ell}(V)$ is defined by
\[
(f \otimes g) (v_1, \dotsc, v_{k+\ell}) = f(v_1,\dotsc,v_k)g(v_{k+1},\dotsc,v_{k+\ell}).
\]

\subsubsection{Bilear map as a tensor product [Example \ref{bilinear-map-as-a-tensor-product}]}\label{bilinear-map-as-a-tensor-product}
Let $e_1,\dotsc,e_n$ be a basis for a vector space $V$, $\alpha^1,\dotsc,\alpha^n$ the dual basis in $V^*$, and $\langle\ ,\ \rangle \colon V \times V \to \bb{R}$ a bilinear map on $V$. Then,
\[
\langle\ , \ \rangle = \sum g_{ij} \alpha^i \otimes \alpha^j,
\]
where $g_{ij} = \langle e_i, e_j \rangle$.

\subsubsection{Wedge product of two multilinear functions [Definition \ref{wedge-product-of-two-multilinear-functions}]}\label{wedge-product-of-two-multilinear-functions}
For $f \in A_k(V), g \in A_\ell(V)$ on a vector space $V$, their \EMPH{wedge product} or \EMPH{exterior product} is
\begin{align*}
f \wedge g &= \fr1{k!\, \ell!} A( f \otimes g).
\end{align*}
$f \wedge g$ is alternating.

Explicitly,
\begin{align*}
(f \wedge g) (v_1,\dotsc,v_{k+\ell}) &= \fr1{k!\,\ell!} \sum_{\sigma \in \g S_{k+\ell}} (\sgn \sigma)f(v_{\sigma(1)},\dotsc,v_{\sigma(k)}) g(v_{\sigma(k+1)},\dotsc,v_{\sigma(k+\ell)}) \\
&= \sum_{\substack{(k,\ell)\text{-shuffle}\\ \sigma}} (\sgn \sigma) f(v_{\sigma(1)},\dotsc,v_{\sigma(k)}) g(v_{\sigma(k+1)},\dotsc,v_{\sigma(k+\ell)}),
\end{align*}
where a $(k,\ell)$-shuffle means $\sigma(1)<\dotsb<\sigma(k)$ and $\sigma(k+1)<\dotsb < \sigma(k+\ell)$.

\subsubsection{Wedge product is anticommutative [Proposition \ref{wedge-product-is-anticommutative}]}\label{wedge-product-is-anticommutative}
For $f \in A_k (V), g \in A_\ell(V)$ on a vector space $V$,
\[
f \wedge g = (-1)^{k\ell} g \wedge f.
\]

If the degree of $f$ is odd, then $f \wedge f = 0$.

\subsubsection{Properties of nesting alternating operators [Lemma \ref{properties-of-nesting-alternating-operators}]}\label{properties-of-nesting-alternating-operators}
For a $k$-tensor $f$ and $\ell$-tensor $g$ on a vector space $V$,
\begin{enumerate}
\item $A(A(f)\otimes g) = k!\,A(f\otimes g)$,
\item $A(f\otimes A(g)) = \ell!\,A(f\otimes g)$.    
\end{enumerate}

\subsubsection{Associativity of the wedge product [Proposition \ref{associativity-of-the-wedge-product}]}\label{associativity-of-the-wedge-product}
For $f \in A_k(V), g \in A_\ell(V), h \in A_m(V)$ on a real vector space $V$,
\[
(f \wedge g) \wedge h = f \wedge ( g \wedge h).
\]

Similarly, for $f_i \in A_{d_i}(V)$ $(i = 1, \dotsc, r)$,
\[
f_1 \wedge \dotsb \wedge f_r = \fr1{(d_1)!\dotsm (d_r)!} A(f_1 \otimes \dotsm \otimes f_r).
\]

\subsubsection{Wedge product of covectors is the determinant [Proposition \ref{wedge-product-of-covectors-is-the-determinant}]}\label{wedge-product-of-covectors-is-the-determinant}
For covectors $\alpha^1, \dotsc , \alpha^k$ on a vector space $V$,
\[
(\alpha^1\wedge \dotsm \wedge \alpha^k)(v_1,\dotsc, v_k) = \det (\alpha^i (v_j))_{ij}.
\]

\subsubsection{Graded algebra over a field [Definition \ref{graded-algebra-over-a-field}]}\label{graded-algebra-over-a-field}
An algebra $\bb{A}$ over a field $\bb{K}$ is said to be \EMPH{graded} if $\bb{A} = \bigoplus_{k = 0}^\infty A^k$ is a direct sum of vector spaces over $\bb{K}$ such that the multiplication sends $A^k \times A^l$ to $A^{k+l}$. $A = \bigoplus_{k = 0}^\infty A^k$ means each nonzero $a \in \bb{A}$ is uniquely a finite sum $a = a_{i_1} + \dotsb a_{i_m}$ where nonzero $a_{i_j} \in A^{i_j}$.

$\bb{A}$ is \EMPH{anticommutative} or \EMPH{graded commutative} if $\forall a \in A^k, b \in A^\ell$, $ab = (-1)^{k\ell} ba$.

A \EMPH{homomorphism} of graded algebras is an algebra homomorphism that preserves the degree.

\subsubsection{Grassmann algebra of multicovectors on a vector space [Definition \& Proposition \ref{grassmann-algebra-of-multicovectors-on-a-vector-space}]}\label{grassmann-algebra-of-multicovectors-on-a-vector-space}
For a vector space $V$ of degree $n < \infty$, the \EMPH{exterior algebra} or the \EMPH{Grassmann algebra} of multicovectors on $V$ is the anticommutative graded algebra
\[
A_* (V) = \bigoplus_{k = 0}^\infty A_k(V) = \bigoplus_{k = 0}^n A_k(V)
\]
with the wedge product of multicovectors as multiplication.

\subsubsection{Wedge product of the dual basis applying to a basis [Lemma \ref{wedge-product-of-the-dual-basis-applying-to-a-basis}]}\label{wedge-product-of-the-dual-basis-applying-to-a-basis}
Let $e_1, \dotsc, e_n$ be a basis for a vector space $V$ and $\alpha^1, \dotsc, \alpha^n$ the dual basis in $V^*$. For $I = (i_1, \dotsc, i_k), J = (j_1, \dotsc, j_k)$ with $1 \le i_1 < \dotsb < i_k \le n,\ 1 \le j_1 < \dotsb < j_k \le n$,
\[
\alpha^I (e_J) = \delta^I_J.
\]

\subsubsection{Wedge products of the dual basis form a basis for multicovectors [Proposition \ref{wedge-products-of-the-dual-basis-form-a-basis-for-multicovectors}]}\label{wedge-products-of-the-dual-basis-form-a-basis-for-multicovectors}
Let $V$ be a vector space and $\alpha^1,\dotsc,\alpha^n$ the dual basis in $V^*$. Then, $\alpha^I$, $I = (i_1 < \dotsb < i_k)$ form a basis for $A_k(V)$.

Therefore,
\[
\dim A_k(V) = \binom{n}{k},
\]
which implies
\[
\text{if}\ k > \dim V,\ \text{then}\ A_k(V) = 0.
\]

\subsubsection{Cotangent space to an Euclidean space at a point [Definition \ref{cotangent-space-to-an-euclidean-space-at-a-point}]}\label{cotangent-space-to-an-euclidean-space-at-a-point}
The \EMPH{cotangent space} to $\bb{R}^n$ at $p$ is $T^*_p (\bb{R}^n) = (T_p(\bb{R}^n))^*$.

\subsubsection{Differential 1-form on an open subset of an Euclidean space [Definition \ref{differential-1-form-on-an-open-subset-of-an-euclidean-space}]}\label{differential-1-form-on-an-open-subset-of-an-euclidean-space}
A \EMPH{covector field} or a \EMPH{differential 1-form} on $U \in \usualtop$ is $\omega \colon U \to \bigcup_{p \in U} T^*_p (\bb{R}^n)$ that maps $U \ni p \mapsto \omega_p \in T^*_p (\bb{R}^n)$.

\subsubsection{Differential of a smooth function [Definition \ref{differential-of-a-smooth-function}]}\label{differential-of-a-smooth-function}
For $f \in C^\infty (U)$ on $U \in \usualtop$, the \EMPH{differential} $df$ of $f$ is a differential 1-form defined by
\[
(df)_p (X_p) = X_p f.
\]


In the expression
\[
\langle \ ,\  \rangle \colon T_p(\bb{R}^n) \times C^\infty_p (\bb{R}^n) \ni (X_p, f) \mapsto \langle X_p, f \rangle = X_p f \in \bb{R},
\]
a tangent vector is considered as $\langle X_p, \cdot \rangle$; a differential at $p$ as $df|_p = (df)_p = \langle \cdot, f \rangle$.


\subsubsection{Differentials of coordinates is the dual basis for the cotangent space [Proposition \ref{differentials-of-the-coordinates-is-the-dual-basis-for-the-cotangent-space}]}\label{differentials-of-the-coordinates-is-the-dual-basis-for-the-cotangent-space}
For $p \in \bb{R}^n$, $\{ (dx^1)_p, \dotsc, (dx^n)_p \} $ is the dual basis for $T_p^* (\bb{R}^n)$ to $\{ \partial / \partial x^1|_p, \dotsc, \partial/\partial x^n|_p \} \subset T_p (\bb{R}^n)$, where $x^1, \dotsc, x^n$ are the standard coordinates on $\bb{R}^n$.

For any differential 1-form $\omega$ on $U \in \usualtop$ and $p \in U$,
\[
\omega_p = \sum a_i(p) (dx^i)_p
\]
for some $a_i (p)$. In this case, $\omega$ is written as $\omega = \sum a_i dx^i$.

\subsubsection{Smoothness of a differential 1-form [Definition \ref{smoothness-of-a-differential-1-form}]}\label{smoothness-of-a-differential-1-form}
A differential 1-form $\omega = \sum a_i dx^i$ on $U \in \usualtop$ is \EMPH{smooth} if all $a_i \colon U \to \bb{R}$ are smooth.

\subsubsection{Differentials can be written in terms of partial derivatives [Proposition \ref{differentials-can-be-written-in-terms-of-partial-derivatives}]}\label{differentials-can-be-written-in-terms-of-partial-derivatives}
For $f \in C^\infty(U)$ on $U \in \usualtop$,
\[
df = \sum \pfr{f}{x^i} dx^i.
\]
Smoothness of $f$ implies that of $df$.

\subsubsection{Differential $k$-forms on an Euclidean space [Definition \ref{differential-k-forms-on-an-euclidean-space}]}\label{differential-k-forms-on-an-euclidean-space}
A \EMPH{differential k-form} or \EMPH{differential form of degree k} on $U \in \usualtop$ is $\omega \colon U \ni p \mapsto \omega_p \in A_k (T_p(\bb{R}^n))$.

\subsubsection{Basis for differential forms [Definition \& Proposition \ref{basis-for-differential-forms}]}\label{basis-for-differential-forms}
Since $\{ dx^I_p \mid I = (1 \le i_1 < \dotsb < i_k \le n) \}$ is a basis for $A_k (T_p(\bb{R}^n)$, for a differential $k$-form $\omega$ on $U \in \usualtop$ and $p \in U$,
\[
\omega_p = \sum a_I (p) dx^I_p,\quad \omega = \sum a_I dx^I.
\]
$\omega$ is \EMPH{smooth} if all $a_I \colon U \to \bb{R}$ are smooth. The vector space of $C^\infty$  differential $k$-forms on $U$ is denoted by $\Omega^k(U)$. If $k = 0$, $\Omega^0(U) = C^\infty(U)$.

\subsubsection{Wedge product of differential forms [Definition \ref{wedge-product-of-differential-forms}]}\label{wedge-product-of-differential-forms}
For differential $k$-form $\omega$ and $\ell$-form $\tau$ on $U \in \usualtop$, their \EMPH{wedge product} $\omega \wedge \tau$ is a differential $(k+\ell)$-form defined by
\[
(\omega \wedge \tau)_p = \omega_p \wedge \tau_p.
\]
If $\omega = \sum a_I dx^I$, $\tau = \sum b_J dx^J$,
\begin{align*}
\omega \wedge \tau &= \sum_{I,J} (a_I b_J) dx^I \wedge dx^J \\
&= \sum_{\text{disjoint}\ I, J} (a_I b_J) dx^I \wedge dx^J.
\end{align*}
For $\omega \in \Omega^k (U)$, $\tau \in \Omega^\ell (U)$, the wedge product is a bilinear map
\[
\wedge \colon \Omega^k(U) \times \Omega^\ell(U) \to \Omega^{k+\ell}(U).
\]
In particular, if $f \in C^\infty (U)$ and $\omega \in \Omega^k(U)$, then $f \wedge \omega = f \omega$.

\subsubsection{Graded algebra with smooth differential forms [Definition \ref{graded-algebra-with-smooth-differential-forms}]}\label{graded-algebra-with-smooth-differential-forms}
For $U \in \usualtop$, the direct sum $\Omega^* (U) = \bigoplus_{k = 0}^n \Omega^k(U)$ is an anticommutative graded algebra over $\bb{R}$ with the wedge product as multiplication, which is also a module over $C^\infty (U)$.


\subsubsection{Differential forms as linear maps on a vector field [Definition \ref{differential-forms-as-linear-maps-on-a-vector-field}]}\label{differential-forms-as-linear-maps-on-a-vector-field}
For a differential $k$-form $\omega$ on $U \in \usualtop$ and $X_1, \dotsc, X_k \in \g X(U)$, define $\omega (X_1, \dotsc, X_k) \in C^\infty(U)$ by
\[
(\omega (X_1, \dotsc, X_k))_p = \omega_p ((X_1)_p, \dotsc, (X_k)_p).
\]
The map
\[
\g X(U) \times \dotsm \times \g X(U) \ni (X_1, \dotsc, X_k) \mapsto \omega(X_1, \dotsc, X_k) \in C^\infty(U)
\]
is $k$-linear over $C^\infty(U)$.

\subsubsection{Exterior derivatives of differential forms [Definition \ref{exterior-derivatives-of-differential-forms}]}\label{exterior-derivatives-of-differential-forms}
For $k \ge 1$ and $\omega = \sum a_I dx^I \in \Omega^k(U)$, the \EMPH{exterior derivative} of $\omega$ is
\[
d\omega = \sum_I da_I \wedge dx^I = \sum_{I,j} \pfr{a_I}{x^j} dx^j \wedge dx^I \in \Omega^{k+1}(U);
\]
for $k = 0$ and $f \in C^\infty(U)$, its exterior derivative is
\[
df = \sum \pfr{f}{x^i} dx^i\in \Omega^1 (U).
\]

\subsubsection{Antiderivation of a graded algebra [Definition \ref{antiderivation-of-a-graded-algebra}]}\label{antiderivation-of-a-graded-algebra}
An \EMPH{antiderivation} of a graded algebra $\bb A = \bigoplus_{k=0}^\infty A^k$ is a linear map $D \colon \bb A \to \bb A$ such that for $a \in A^k, b \in A^\ell$,
\[
D(ab) = D(a)b + (-1)^k aD(b).
\]
If $m$ is an integer such that $D$ sends $A^k$ to $A^{k+m}$ for all $k$, then $m$ is called the \EMPH{degree} of $D$.

\subsubsection{Properties of the exterior differentiation [Proposition \ref{properties-of-the-exterior-differentiation}]}\label{properties-of-the-exterior-differentiation}
\begin{enumerate}
\item The exterior differentiation $d \colon \Omega^*(U) \to \Omega^*(U)$ on $U \in \usualtop$ is an antiderivation of degree 1:
\[
d(\omega \wedge \tau) = (d\omega)\wedge \tau + (-1)^{\deg \omega} \omega \wedge d \tau.
\]
\item $d^2 = 0$.
\item For $f \in C^\infty(U)$ and $X \in \g X(U)$, $(df)(X) = Xf$.
\end{enumerate}

\subsubsection{Characterization of the exterior differentiation [Proposition \ref{characterization-of-the-exterior-differentiation}]}\label{characterization-of-the-exterior-differentiation}
The exterior differentiation $d \colon \Omega^*(U) \to \Omega^*(U)$ on $U \in \usualtop$ is the only antideriavtion of $\Omega^*(U)$.
