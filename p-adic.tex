\section{P-adic Numbers}\cite{gouvea}
\subsection{Foundations}
\DEF{Absolute value on a field,absolute-value-on-a-field}
An \EMPH{absolute value} on a field $\bb{K}$ is a function $\mid \ \ \mid \colon \bb{K} \to \bb{R}_{\ge0}$ that satisfies:
\begin{enumerate}
\item $\abs{x} = 0$ iff $x = 0$
\item $\forall x, y \in \bb{K},\ \abs{xy} = \abs{x}\abs{y}$
\item $\forall x, y \in \bb{K},\ \abs{x+y} \le \abs{x}+\abs{y}$.
\end{enumerate}

An absolute value that satisfies the condition
\begin{enumerate}[resume*]
\item $\forall x, y \in \bb{K},\ \abs{x+y} \le \max \{ \abs{x},\abs{y} \}$
\end{enumerate}
is said to be \EMPH{non-archimedean}; otherwise, it is said to be \EMPH{archimedean}.

\DEF{Trivial absolute value,trivial-absolute-value}
The \EMPH{trivial absolute value} on a field $\bb{K}$ is a absolute value on $\bb{K}$ such that
\[
\abs{x} = \begin{cases}
 1 & \text{for}\ x \neq 0 \\
 0 & \text{for}\ x = 0
 \end{cases}
\]

An absolute value on a finite field must be trivial.

\DEF{Valuation on a field,valuation-on-a-field}
A function $v \colon \bb{A}^\times \to \bb{R}$ with an integral domain $\bb{A}$ is called a \EMPH{valuation} on $\bb{A}$ if it satisfies the following conditions:
\begin{enumerate}
\item $\forall x, y \in \bb{A}^\times,\ v(xy) = v(x) + v(y)$
\item $\forall x, y \in \bb{A}^\times,\ v(x+y) \ge \min \{ v(x), v(y) \}$
\end{enumerate}

\DPROP{Value group of a valuation,value-group-of-a-valuation}
The image of a valuation $v$ on a field is an additive subgroup of $\bb{R}$. $\im v$ is called the \EMPH{value group} of $v$.

\PROP{Correspondence between valuations and non-archimedean absolute values,correspondence-between-valuations-and-non-archimedean-absolute-values}
Let $\bb{A}$ be an integral domain and $\bb{K} = \enfrac\bb{A}$. Let $v\colon \bb{A}^\times \to \bb{R}$ be a valuation on $\bb{A}$ and extend $v$ to $\bb{K}$ by setting $v(a/b) = v(a) - v(b)$, then the function $\mid\ \ \mid_v \colon \bb{K} \to \bb{R}_{\ge 0}$ defined by
\[
\abs{x}_v = \begin{cases}
 e^{-v(x)} & \text{for}\ x \neq 0 \\
 0 & \text{for}\ x = 0
 \end{cases}
 \] is a non-archimedean absolute value on $\bb{K}$. Conversely, $-\log \mid \,\ \mid$ is a valuation on $\bb{K}$ for a non-archimedean absolute value $\mid\ \ \mid$ on $\bb{K}$.


\DEF{p-adic valuation,p-adic-valuation}
The $p$\EMPH{-adic valuation} on $\bb{Q}$ with a prime $p$ is a valuation $v_p \colon \bb{Q}^\times \to \bb{R}$ defined as follows: for each $n \in \bb{Z}^\times$, let $v_p(n)$ be the greatest integer such that $p^{v_p(n)} \mid n$, and for each $x = a / b \in \bb{Q}^\times$, $v_p(x) = v_p(a) - v_p(b)$.

We often set $v_p(0) = \infty$.

\DEF{p-adic absolute value,p-adic-absolute-value}
The $p$\EMPH{-adic absolute value} $\mid\ \ \mid_p \colon \bb{Q} \to \bb{R}_{\ge0}$ with a prime $p$ is defined as
\[
\abs{x}_p = p^{-v_p(x)},\quad |0| = 0.
\]

The usual absolute value is looked as $\mid\ \ \mid = \mid\ \ \mid_\infty$.

\DEF{Absolute values on a field of rational functions,absolute-values-on-a-field-of-rational-functions}

Here are some absolute values on a field $\bb F(t)$ of rational functions over a field $\bb F$.

\begin{enumerate}
\item For $f(t) \in \bb F[t]$, $v_\infty (f) = -\deg f$, and for $f(t)/g(t) \in \bb F(t)$, $v_\infty (f/g) = v_\infty(f) - v_\infty (g)$ with $v_\infty (0) = \infty$. Then,
\[
\abs{f(t)}_\infty = e^{-v_\infty (f)}.
\]
\item For an irreducible polynomial $p(t) \in \bb F[t]$, define the $p(t)$-adic valuation and absolute value.
\end{enumerate}

\LEM{Properties of absolute values on fields,properties-of-absolute-values-on-fields}
For an absolute value $|\ \ |$ on a field $\bb K$,
\begin{enumerate}
\item $\abs{1} = 1$,
\item $\forall x \in \bb K,\ \abs{x^n} = 1 \Rightarrow \abs x = 1$,
\item $\forall x \in \bb K,\ \abs{-x} = \abs x$,
\item If $\bb K$ is finite, then $|\ \ |$ is trivial.
\end{enumerate}

\THM{Necessary and sufficient conditions of a non-archimedean absolute value,necessary-and-sufficient-conditions-of-a-non-archimedean-absolute-value}

Let $\bb K$ be a field, $|\ \ |$ an absolute value on $\bb K$. Then,
\begin{align*}
|\ \ |\ \text{is non-archimedean} &\Longleftrightarrow \forall n = 1 + \dotsb + 1 \in \bb K,\ \abs n \le 1 \\
&\Longleftrightarrow \sup \{ \abs n \mid n \in \bb Z \} = 1.
\end{align*}

Furthermore, $\sup \{ \abs n \mid n \in \bb Z \} = \infty$ if $|\ \ |$ is archimedean. 

\DEF{Distance induced from an absolute value on a field,distance-induced-from-an-absolute-value-on-a-field}
Given a field $\bb K$ with an absolute value $|\ \ |$, define the distance $d \colon \bb K \times \bb K \to \bb R$ by $d(x,y) = \abs{x-y}$, which makes $\bb K$ into a metric space.

\PROP{Field can be a topological field with the induced distance,field-can-be-a-topological-field-with-the-induced-distance}
For a field $\bb K$ with the distance $d \colon \bb K \times \bb K \to \bb R$ induced from an absolute value on $\bb K$, addition, multiplication and taking inverse are all continuous: fix points $x_0, y_0 \in \bb K$, then
\begin{itemize}
  \item $\forall \epsilon > 0, \exists \delta > 0, \forall x, y \in \bb K,\ d(x, x_0), d(y, y_0) < \delta \Longrightarrow d(x+y, x_0 + y_0) < \epsilon$,
  \item $\forall \epsilon > 0, \exists \delta > 0, \forall x, y \in \bb K,\ d(x, x_0), d(y, y_0) < \delta \Longrightarrow d(x-y, x_0 - y_0) < \epsilon$,
  \item $\forall \epsilon > 0, \exists \delta > 0, \forall x \in \bb K,\ d(x, x_0) < \delta \Longrightarrow d(\fr1x, \fr1{x_0}) < \epsilon$.
\end{itemize}

\DPROP{Ultrametric space with a non-archimedean absolute value,ultrametric-space-with-a-non-archimedean-absolute-value}
For a field $\bb K$ with the distance $d$ induced from an absolute value $|\ \ |$ on $\bb K$, $|\ \ |$ is non-archimedean iff $d(x,y) \le \max\{ d(x,z), d(z,y)\}$. This inequality is called the \EMPH{ultrametric inequality}, a metric which holds the ultrametric inequality an \EMPH{ultrametric}, and a space with an ultrametric an \EMPH{ultrametric space}.

\PROP{Non-archimedean absolute value of a sum is the max of the two,non-archimedean-absolute-value-of-a-sum-is-the-max-of-the-two}
For a field $\bb K$ with a non-archimedean absolute value $|\ \ |$ on $\bb K$, and $x, y \in \bb K$ with $\abs x \neq \abs y$,
\[
  \abs{x+y} = \max\{ \abs x, \abs y \}.
\]

\COR{Triangle on an ultrametric space is isosceles,triangle-on-an-ultrametric-space-is-isosceles}
In an ultrametric space, all triangles are isosceles.

\DEF{Open ball and closed ball on a field with an absolute value,open-ball-and-closed-ball-on-a-field-with-an-absolute-value}
In a field $\bb K$ with an absolute value $|\ \ |$, an \EMPH{open ball} of radius $r > 0$ and center $a \in \bb K$ is
\[
  B(a, r) = \{ x \in \bb K \mid \abs{x - a} < r \};
\]
a \EMPH{closed ball} is
\[
  \BAR B(a, r) = \{ x \in  \bb K \mid \abs{x - a} \le r \}.
\]

\PROP{Properties of an open ball and closed ball with a non-archimedean absolute value,properties-of-an-open-ball-and-closed-ball-with-a-non-archimedean-absolute-value}
Let $\bb K$ be a field with a non-archimedean absolute value, and $a, b \in \bb K,\ r, s > 0$.
\begin{enumerate}
  \item $b \in B(a, r) \Longrightarrow B(a, r) = B(b, r)$,
  \item $b \in \BAR B(a, r) \Longrightarrow \BAR B(a, r) = \BAR B(b, r)$,
  \item $B(a, r)$ is clopen,
  \item $\BAR B(a, r)$ is clopen,
  \item $B(a, r) \cap B(b, s) \Longleftrightarrow B(a, r) \subset B(b, s) \ \text{or}\ B(a, r) \supset B(b, s)$.
  \item $\BAR B(a, r) \cap \BAR B(b, s) \Longleftrightarrow \BAR B(a, r) \subset \BAR B(b, s) \ \text{or}\ \BAR B(a, r) \supset \BAR B(b, s)$.
\end{enumerate}
