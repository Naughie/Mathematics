\makeatletter
\def\@cite#1{\textsuperscript{\,[#1]}}
\def\@biblabel#1{[#1]}
\def\includeall#1{{%
\@for\next:=#1\do{\include{\next}}
}}
\def\my@enparen#1#2;{%
  \ifx\relax#2\relax%
    #1%
  \else%
    (#1#2)%
  \fi}
%\newcommand{\myfrac}{\@ifstar{\@myfrac}{\@@myfrac}}
%\newcommand{\@myfrac}[2]{\mathchoice{\frac{#1}{#2}}{#1/#2}{#1/#2}{#1/#2}}
%\newcommand{\@@myfrac}[2]{\mathchoice{\frac{#1}{#2}}{\my@enparen#1;/\my@enparen#2;}{\my@enparen#1;/\my@enparen#2;}{\my@enparen#1;/\my@enparen#2;}}
\makeatother
\ifupTeX
\renewcommand{\rmdefault}{cmbr}
\renewcommand{\sfdefault}{phv}
\newcommand{\myemph}{\fontfamily{ptm}\fontseries{b}\fontshape{it}\selectfont}
\else\ifluatex
\setmainfont[Numbers = Uppercase,
             Ligatures = TeX,
             BoldFont  = CMU Bright SemiBold,
             ItalicFont = CMU Bright Oblique,
            ]{CMU Bright Roman}
\setsansfont[Ligatures=TeX]{TeXGyreHeros}
\newcommand{\myemph}{\setmainfont{Times New Roman Bold Italic}}
\else\ifxetex
\setmainfont[
    Extension=.otf,
    UprightFont=*mr,
    ItalicFont=*mo,
    BoldFont=*sr,
    BoldItalicFont=*so,
    NFSSFamily=cmbr
    ]{cmunb}
\setsansfont{Helvetica}
\newcommand{\myemph}{\setmainfont{Times New Roman Bold Italic}}
\fi\fi\fi
\newcommand{\empha}[1]{{\myemph #1}}
\newcommand{\roma}[1]{{\fontfamily{lmr}\selectfont #1\relax}}
\renewcommand{\familydefault}{\rmdefault}
\newcommand{\myTitle}{{\roma{Notes of Mathematics}}}
\newcommand{\myAuthor}{{\roma{Masato Nakata}}}
\newcommand{\myAffil}{{\roma{Department of Science, Kyoto University}}}
\newcommand{\myDate}{{\roma{Since Aug 27, 2017}}}
\renewcommand{\refname}{References}
\renewcommand{\contentsname}{Contents}
\renewcommand{\labelenumi}{\roman{enumi})}
\newtheorem{theorem}{Theorem}[section]
\newtheorem{lemma}{Lemma}[section]
\newtheorem{proposition}{Proposition}[section]
\theoremstyle{definition}
\newtheorem{definition}{Definition}[section]
\newcommand{\bb}[1]{\mathbb{#1}}
\newcommand{\fr}[2]{\mathchoice{\frac{#1}{#2}}{#1/#2}{#1/#2}{#1 / #2}}
\newcommand{\pfr}[2]{\fr{\partial #1}{\partial #2}}
\newcommand{\abs}[1]{\left|  #1 \right|}
\newcommand{\g}[1]{\mathfrak{#1}}
\newcommand{\s}[1]{\mathscr{#1}}
\newcommand{\usualtop}[1][n]{\mathcal{O}_{#1}}
\DeclareMathOperator{\im}{im}
\DeclareMathOperator{\tr}{tr}
\DeclareMathOperator{\sgn}{sgn}
\DeclareMathOperator{\ad}{ad}
\DeclareMathOperator{\der}{Der}