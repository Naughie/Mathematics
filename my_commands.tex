\makeatletter
\def\@cite#1{\textsuperscript{\,[#1]}}
\def\@biblabel#1{[#1]}
\def\includeall#1{{%
\@for\next:=#1\do{\include{\next}}
}}
\makeatother
\renewcommand{\familydefault}{cmbr}
\renewcommand{\sfdefault}{phv}
\renewcommand{\rmdefault}{lmr}
\newcommand{\empha}[1]{{\fontfamily{ptm}\fontseries{b}\fontshape{it}\selectfont #1}}
\newcommand{\myTitle}{{\rmfamily Notes of Mathematics}}
\newcommand{\myAuthor}{{\rmfamily Masato Nakata}}
\newcommand{\myAffil}{{\rmfamily Department of Science, Kyoto University}}
\newcommand{\myDate}{{\rmfamily Since Aug 27, 2017}}
\renewcommand{\refname}{References}
\renewcommand{\contentsname}{Contents}
\renewcommand{\labelenumi}{\roman{enumi})}
\newtheorem{theorem}{Theorem}[section]
\newtheorem{lemma}{Lemma}[section]
\newtheorem{proposition}{Proposition}[section]
\theoremstyle{definition}
\newtheorem{definition}{Definition}[section]
\newcommand{\bb}[1]{\mathbb{#1}}
\newcommand{\fr}[2]{#1 / #2}\everydisplay{\let\fr\frac}
\newcommand{\pfr}[2]{\fr{\partial #1}{\partial #2}}
\newcommand{\abs}[1]{\left|  #1 \right|}
\newcommand{\g}[1]{\mathfrak{#1}}
\newcommand{\s}[1]{\mathscr{#1}}
\newcommand{\usualtop}[1][n]{\mathcal{O}_{#1}}
\DeclareMathOperator{\im}{im}
\DeclareMathOperator{\tr}{tr}
\DeclareMathOperator{\sgn}{sgn}